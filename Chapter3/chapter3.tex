\chapter{My third chapter}

% **************************** Define Graphics Path **************************
\ifpdf
    \graphicspath{{Chapter3/Figs/Raster/}{Chapter3/Figs/PDF/}{Chapter3/Figs/}}
\else
    \graphicspath{{Chapter3/Figs/Vector/}{Chapter3/Figs/}}
\fi

\section{First section of the third chapter}
And now I begin my third chapter here \dots

And now to cite some more people~\citet{Rea85,Ancey1996}

\subsection{First subsection in the first section}
\dots and some more 

\subsection{Second subsection in the first section}
\dots and some more \dots

\subsubsection{First subsub section in the second subsection}
\dots and some more in the first subsub section otherwise it all looks the same
doesn't it? well we can add some text to it \dots

\subsection{Third subsection in the first section}
\dots and some more \dots

\subsubsection{First subsub section in the third subsection}
\dots and some more in the first subsub section otherwise it all looks the same
doesn't it? well we can add some text to it and some more and some more and
some more and some more and some more and some more and some more \dots

\subsubsection{Second subsub section in the third subsection}
\dots and some more in the first subsub section otherwise it all looks the same
doesn't it? well we can add some text to it \dots

\section{Second section of the third chapter}
and here I write more \dots

\section{The layout of formal tables}

You will not go far wrong if you remember two simple
guidelines at all times:

\begin{enumerate}
  \item Never, ever use vertical rules.
  \item Never use double rules.
\end{enumerate}

These guidelines may seem extreme but I have
never found a good argument in favour of breaking them. For
example, if you feel that the information in the left half of
a table is so different from that on the right that it needs
to be separated by a vertical line, then you should use two
tables instead. Not everyone follows the second guideline:
I have worked for a publisher who insisted on a
double light rule above a row of totals. But this would not
have been my choice.

There are three further guidelines worth mentioning here as they
are generally not known outside the circle of professional
typesetters and subeditors:

\begin{enumerate}\setcounter{enumi}{2}
  \item Put the units in the column heading (not in the body of
          the table).
  \item Always precede a decimal point by a digit; thus 0.1
      {\em not} just .1.
  \item Do not use `ditto' signs or any other such convention to
      repeat a previous value. In many circumstances a blank
      will serve just as well. If it won't, then repeat the value.
\end{enumerate}

Whether or not you wish to follow the minor niceties,
if you use only the following commands in your formal tables
your reader will be grateful. I stress that
the guidelines are not just to
keep the pedantic happy. The principal is that enforced structure of
presentation
enforces structured thought in the first instance.

Now we can refer to the table using Table.~\ref{t:borders}.
\begin{table}
\caption{Table with borders}
\centering
\label{t:borders}
\begin{tabular}{l c r}
\toprule
1 & 2 & 3 \\ \midrule
4 & 5 & 6 \\ \midrule
7 & 8 & 9 \\ \bottomrule
\end{tabular}
\end{table}
